\documentclass[7pt]{article}
\raggedright
\parindent=0in \parskip=8pt
\usepackage{graphicx}
\usepackage[margin=1in]{geometry} % 1 inch margins all around
\begin{document}

\begin{Huge}
\begin{center}
\begin{normalsize}
\textbf{MAKERERE \includegraphics[scale=0.5]{logo} UNIVERSITY }\\

\textbf{FACULTY OF COMPUTING AND INFORMATICS TECHNOLOGY} \\
\textbf{SCHOOL OF COMPUTING AND INFORMATICS TECHNOLOGY} \\
\textbf{DEPARTMENT OF COMPUTER SCIENCE} \\
\textbf{BACHELOR OF SCIENCE IN COMPUTER SCIENCE} \\
\textbf{YEAR 2} \\
\textbf{BIT 2207 RESEARCH METHODOLOGY} \\
\textbf{Course Work: Assignment 4}\\
\end{normalsize}
\end{center}
\end{Huge}

\begin{center}
\begin{tabular}{l l l}
\textbf{NAME}  & \textbf{REGISTRATION NUMBER} & \textbf{STUDENT NUMBER} \\
NALUTAAYA VIOLA& 16/U/9075/PS & 216014557 \\
\end{tabular}

\paragraph{•}
\textbf{Lecturer}: Mr. Earnest Mwebaze
\end{center}

\newpage

\title{THE EFFECT OF THE WEB ON RESEARCH}
\author{NALUTAAYA VIOLA}      
\renewcommand{\today}{}

\maketitle

\section{INTRODUCTION}
\paragraph{•}
The web is basically a system of Internet servers that support specially formatted documents. The documents are formatted in a markup language called HTML (Hypertext Markup Language) that supports links to other documents, as well as graphics, audio, and video files. This means you can jump from one document to another simply by clicking on hot spots. Not all Internet servers are part of the World Wide Web.\cite{beal}
\paragraph{•}
Several applications called web browser web make it easy to access the web most commonly used being Firefox and Microsoft internet explorer

\section{LITERATURE REVIEW}
\paragraph{•}
In early 1990s numerous academic libraries adopted the web as communication tool with users.
\paragraph{•}
Early studies typically focused on design characteristics since websites initially merely provided information on the services and collections available in the university’s physical library.
\paragraph{•}
But later 1990s technological developments coupled with new digitization efforts offered new opportunities for websites with commercial and local databases electrical journals e-books and virtual reference.
\paragraph{•}
The availability of new content and services on library websites facilitated research efforts comparing these features among academic library websites.\cite{barbara}

\begin{thebibliography}{9}
\bibitem{beal} Vangie Beal. \textit{Web - World Wide Web}, Internet: www.webopedia.com/TERM/W/World\textunderscore Wide\textunderscore Web.html, [March 8, 2018].
\bibitem{barbara} Barbara A.Blummer. \textit{A Literature Review of Academic Library Web Page Studies} Internet https://www.tandfonline.com/doi/abs/10.1300/J502v01n01\textunderscore 04, 12 Oct 2008 [March 8, 2018].
\end{thebibliography}

\end{document}